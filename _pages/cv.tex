\documentclass[a4paper,9pt]{extarticle}

% Packages
\usepackage[utf8]{inputenc} % For input encoding
\usepackage{geometry} % For page margins
\geometry{a4paper, margin=0.75in} % Set paper size and margins
\usepackage{titlesec} % For section title formatting
\usepackage{enumitem} % For itemized list formatting
\usepackage{hyperref} % For hyperlinks

% Formatting
\setlist{noitemsep} % Removes item separation
\titleformat{\section}{\large\bfseries}{\thesection}{1em}{}[\titlerule] % Section title format
\titlespacing*{\section}{0pt}{\baselineskip}{\baselineskip} % Section title spacing

% Begin document
\begin{document}

% Disable page numbers
\pagestyle{empty}

% Header
\begin{center}
\textbf{\Large Marie-Félicia Beclin}\\[2pt] % Name
Montpellier, France| \href{mailto:mariefelicia.beclin@gmail.com}{mariefelicia.beclin@gmail.com} | 0687470308  \href{https://www.linkedin.com/in/johndoe}{linkedin.com/in/johndoe} | France % Contact info

\end{center}

% Education Section
\section*{EDUCATION}
\noindent
\textbf{University of Montpellier}, Montpellier, France \hfill  10/2021 | 11/2024 (Expected)\\ % University name and location
\textit{PhD Project : Statistical analysis of thoracic computed tomography scanner in case of asthma patient treated by Benralizumab }\\

The objective is to ascertain its efficacy of Benralizumab, a medication to treat asthma. The different steps of the project consist of :
\begin{itemize}
    \item CT Scan segmentation (with two separate techniques : by theresholding with ITK and by neural network "Lungmask") and B-spline registration between inspiration and expiration 
    \item Computation of histogram in 1D and in 2D 
    \item Computation of the PRM, parametric response map
    \item Bibliography and study on regression distribution on distribution ( Optimal transport, Frechet regression )  
\end{itemize}

\textbf{Talks}
\begin{itemize}
\item Soon :  Seminaire IDESP December 2023
\item Soon : Seminaire 
\item "Conférence IA et Santé", Nantes, 2022
\end{itemize}



\noindent
\textbf{Mines ParisTech,}, Paris, France \hfill 2017 
\begin{itemize}
\item Information system management
\item Hardware and software architectures
\item Introduction to artificial intelligence
\item Image Processing
\end{itemize}\\

\noindent
\textbf{Mines Saint-Etienne}, Saint-Etienne, France \hfill 2015-2017 
\begin{itemize}
\item Majeur Data Sciences (Optimisation, Statistics,machine learning,  R and Python coding) 
\item Options : Functional Equations and Geometry - High Performance Calculations - Image Processing
\end{itemize}\\

\noindent
\textbf{Université Jean Monnet},Saint-Etienne, France \hfill 2015 \\
\textit{L3-Mathematiques in parallelle with engineering cursus}\\

\noindent
\textbf{Lycée Louis Le Grand},Paris, France} \hfill 2015-2013 \\
\textit{Prep school - MPSI-MP*} ( Maths, Physics and engeneering sciences)\\

\noindent
\textbf{Lycée Paul Hazard},Armentières, France} \hfill 2013 \\
\textit{Baccalauréat Scientific  with honours}


% Experience Section
\section*{WORK EXPERIENCE}
% Additional Experience or Volunteer Work
\noindent
\textbf{ B12 Consulting - Louvain-La-Neuve, Belgium } \hfill \\ 
\textit{Analyst developper} \hfill 2020 % Position and duration
\begin{itemize}
    \item Data sciences' project (Python , Pandas, Numpy, Scikit-Learn)
    \item Back-end Development (Python - Django et Django REST) et Front - end ( Typescript - React) 
\end{itemize}\\

\noindent
\textbf{Engie - Louvain-La-Neuve, Belgium}\\ % Project or organization name and location
\textit{Data Scientist Junior} \hfill 2017 – 2019 % Position and duration
\begin{itemize}
	\item Internship's project : Study and detection of the wake effect on wind farms using data science algorithms
	\begin{itemize}

\item 1. Bibliographic research on the wake
\item 2. Exploration and analysis of data (weather and performance data on wind farm)
\item 3. Determination and implementation of a model
\item 4. Validation of the model / Tools used: Python, Pandas, Numpy, Scikit-Learn
	\end{itemize}
    \item Data science projects - Studies of degradation rates of solar panels and inverters - Python
Scikit-learn / Pandas
 \item Creation of an automation tool for downloading documents relating to calls for tenders - Python / Selenium % Responsibilities and achievements
\end{itemize}





\end{itemize}

\section*{TEACHING}
\textbf{Teaching at university}
Algèbre Licence 1 ( Espace vectoriel , applications linéaires )
\begin{itemize} 
\item Algebra Licence 1 ( vectorial space , linear applications )
\item Probability for studient in life's science ( Probability on finite univers, binomial, poisson, geomoetric distributions )
\item Prépa polytech first year : Algebra ( vectorial space , linear applications,matrix calculus, basis, Gauss ' pivot)
\item Prépa polytech first year : Analysis (Riemann integrals over a segment $[a,b]$ )
\item Prépa polytech second year : Analysis ( Numerical sequences and series, sequences and series of functions, power series, improper integral )

\end{itemize}



\textbf{Academic support in a social center}
4h30 per week ( 2021-2022) - 1h per week (2023)


% Skills Section
\section*{SKILLS}
\begin{itemize}

    \item \textbf{Programming:} Python: Numpy, Pandas, ITK, Scikit-Learn.
    SQL, R,  % Programming skills
    \item \textbf{Language: } French (Native)
  
\end{itemize}

% End document
\end{document}
